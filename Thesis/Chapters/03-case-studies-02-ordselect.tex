% !TeX root = ../main.tex
% !TeX spellcheck = en_US

\section{Median and Order Statistics}
Our final case study is concerned with computing order statistics. The $k$th order statistic is the $k$th smallest value in a set. The median is a special case of an order statistic with $k = \frac n 2$. The obvious approach to calculating order statistics is to first sort the set and then select the value at index $k$. With current search algorithms, this can be done in time $\mathcal O(n \log n)$. However we can do better: The quickselect algorithm of Hoare \cite{hoare:1961:quickselect} calculates order statistics in linear time. The algorithm works by quickly identifying the top and bottom 30\% of values so in each iteration we can eliminate a significant chunk of the data set.

Often we are not only interested in \emph{what} the $k$th order statistics is, but also where to find it in the original data set, and which values are smaller or larger than it. We calculate this information anyway, so we return it alongside the actual value.

\begin{lstlisting}[caption={The Indexed Type},label={lst:median:indexed},emph={Indexed,index}]
record Indexed (A : Set a) (l : bNat) : Set a where
    constructor index
    field
        idx : Fin l
        value : A
\end{lstlisting}

To store the index of a value, we use Agda's finite number type: the type \texttt{Fin l} contains all natural numbers smaller than $l$. The element \texttt{zero} is part of every \texttt{Fin l}  for $l > 0$ and any other element of some \texttt{Fin (suc l)} is constructed as the successor of a number in \texttt{Fin l}.

\begin{lstlisting}[caption={The Split Type},label={lst:median:split},emph={Split,Indexed}]
record Split (A : Set a) (l : bNat) (b\_1 b\_2 : bNat -> Set) : Set a where
    field
        median : Indexed A l
        {l\_1 l\_2 l\_3} : bNat
        smaller : Vec (Indexed A l) l\_1
        larger : Vec (Indexed A l) l\_2
        unknown : Vec (Indexed A l) l\_3
        length-≡ : suc (l\_1 + l\_2 + l\_3) ≡ l
        bound-smaller : b\_1 l\_1
        bound-larger : b\_2 l\_2
\end{lstlisting}

The \texttt{Split} type, which will form the return type of our median and quickselect functions, is parameterized by four values: The type from which we will draw our values, the size of the set analyzed and two constraint types, indexed by the naturals. The record stores the median value, as well as the remainder of the analyzed set split into a smaller, a larger and a section of which we do not have ordering information.

We guarantee that a \texttt{Split} instance contains the entirety of the originally analyzed set by proxy, by showing that the length of the three subsets plus the median element itself sum up to the original length.

Finally, we have two bounds, one for the size of the smaller and one for the size of the larger subset. This allows us to express the guarantee of the intermediate elimination described above:

\begin{lstlisting}
intermediate :  Vec A l
             -> Split A l (_\geq 3 * (l / 10)) (_\geq 3 * (l / 10))
\end{lstlisting}

As well as that of our final functions such as the median:

\begin{lstlisting}
median :  Vec A l
       -> Split A l (_\equiv \lfloor l /2\rfloor) (_\equiv \lceil l /2\rceil - 1)
\end{lstlisting}


Our intermediate step works as follows: We group our data into sets of five. We can order the elements of each of these sets with respect to their median in constant time.

\begin{lstlisting}[caption={Median of 5},label={lst:median:medofmeds},emph={M,median,by,medians,of}]
M5 : Set a -> bNat -> Set a
M5 A l = Indexed A l × Indexed A l
× Indexed A l
× Indexed A l × Indexed A l

median5-by :  (X -> A)
           -> X -> X -> X -> X -> X
           -> DecTree A (M5 X 5) 7


data Medians-Of-5 (A : Set a) (l : bNat) : Set a where
    medians :  Vec (M5 A l) \lfloor l /5\rfloor
            -> Vec (Indexed A l) (l % 5)
            -> Medians-Of-5 A l

 _:::_ : M5 A 5 -> Medians-Of-5 A l -> Medians-Of-5 A (5 + l)


medians-of-5-by :  (X -> A) -> Vec X l
                -> DecTree A (Medians-Of-5 X l) (2 * l)
medians-of-5-by _ [] =
        return $ medians [] []
medians-of-5-by _ (a :: []) =
        delay 2 $
        return $ medians [] [ index f0 a ]
medians-of-5-by _ (a :: b :: []) =
        delay 4 $
        return $ medians [] $  (index f0 a)
                            :: [ index f1 b ]
medians-of-5-by _ (a :: b :: c :: []) =
        delay 6 $
        return $ medians [] $  (index f0 a)
                            :: (index f1 b)
                            :: [ index f2 c ]
medians-of-5-by _ (a :: b :: c :: d :: []) =
        delay 8 $
        return $ medians [] $  (index f0 a)
                            :: (index f1 b)
                            :: (index f2 c)
                            :: [ index f3 d ]
medians-of-5-by f (a :: b :: c :: d :: e :: xs) =
        let n = len xs in
        height-\equiv (sym $ *-distrib\^l-+ 2 5 n) $
        height-\equiv (cong (10 +_) $ +-identity\^r (2 * n)) $ do
        m5 <- delay 3 $ median5-by f a b c d e
        ms <- medians-of-5-by f xs
        return $ m5 ::: ms

\end{lstlisting}

We ignore the final remainder of the list if it is not divisible by 5, and map our five-element median function over the rest. Each step takes us slightly less than 10 units of time per five elements, and we bound this from above as two times the size of the set.

Having done that, we can order the sets \emph{themselves} by comparing these median values. By transitivity we know that in each set smaller than the median set, the leftmost three items are smaller than the median-of-medians. This also holds symmterically for the larger sets. \todo{Visualize: See Hinze Intro to Programming}

How do we select the median-of-medians? We fall back to the function we wanted to implement originally -- the median selection itself, now on the reduced data set of $\lfloor \frac n 5 \rfloor$.

\begin{lstlisting}[caption={Quasi-Median},label={lst:median:quasimedian},emph={Quasi,Median,quasi,median,by,ordselect,Ordselect}]
Quasi-Median : Set a -> bNat -> Set a
Quasi-Median A l = Split A l
                   (\lambda x -> suc x \geq 3 * (l / 10))
                   (\lambda x -> suc x \geq 3 * (l / 10))

quasi-median-by :  let l = suc l-1 in
                   Acc _<_ l
                -> (X -> A)
                -> Vec X l
                -> DecTree A (Quasi-Median X l) (9 * l)

Ordselect : Set a -> (l : bNat) -> (i : Fin l) -> Set a
Ordselect A l i = Split A l
                  (_\equiv Data.Fin.tobNat i)
                  (_\equiv l Data.Fin.bNat-bNat (Data.Fin.raise 1 i))

ordselect-by :  let l = suc l-1 in
                Acc _<_ l
             -> (X -> A)
             -> (i : Fin l)
             -> (xs : Vec X l)
             -> DecTree A (Ordselect X l i) (35 * l)
\end{lstlisting}

Since our functions are mutually recursive, we have to pass an accessibility proof again. As with our sorting algorithms, we will show termination by way of the length of the vector.

Our elimination function will return a quasi-median: A split that guarantees that guarantees that 30\% (strictly speaking: $3\lfloor l {10}\rfloor$) of elements are smaller and greater than the selected median value. We furthermore guarantee that this is achievable in linear time with constant factor 9.

In a similar vein, our ordselect (\emph{order statistic selection}) algorithm will return a split that guarantees that exactly $i$ elements are smaller than the $i$th order statistic and the remaining $l - i - 1$ elements are larger. This too is achievable in linear time, with constant factor 35.

\begin{lstlisting}[caption={Quasi-Median},label={lst:median:quasimedian}]
quasi-median-by _ _ (a :: []) = delay 9 $ ...
quasi-median-by _ f (a :: b :: []) = delay 17 $ ...
quasi-median-by _ f (a :: b :: c :: []) = delay 24 $ ...
quasi-median-by _ f (a :: b :: c :: d :: []) = delay 31 $ ...
quasi-median-by (Acc.acc more) f
                xs@(_ ∷ _ ∷ _ ∷ _ ∷ _ ∷ xss) =

    let l = suc l-1 in
    height-\equiv (sym $ *-distrib\^r-+ l 2 7) $
    height-\equiv (+-identity\^r (2 * l + 7 * l)) $
    height-\equiv (sym $ +-assoc (2 * l) (7 * l) 0 ) $
    do
        medians ms overflow <- medians-of-5-by f xs
        let ix = ix-half ms
        med-of-meds' <- delay-\leq (a*5*\lfloorn/5\rfloor\leqa*n 7 l) $
                        ordselect-by
                            (more \lfloor l /5\rfloor $ \lfloorl/5\rfloor<l _)
                            (f \circ m5-extract-value) ix ms
        let med-of-meds = simplify-med-split med-of-meds'
        return $ record
            { median = m5-extract-indexed $
                       Indexed.value $
                       Split.median med-of-meds
            ; smaller = small med-of-meds
            ; larger = large med-of-meds
            ; unknown = unk med-of-meds ++ overflow
            ; length-≡ = quasi-median-length-\equiv $ len xss
            ; bound-smaller = \leq-step $ \leq-step $ \leq-step \leq-refl
            ; bound-larger = a+a*n/10\geqa*\lceiln+5\rceil/10 3 (len xss)
            }
  where
    open Data.Fin hiding (_+_ ; _\leq_)
    open import Data.Fin.Properties hiding (\leq-refl)
    open import Fin.Props
    m5-extract-value    :  M5 A l -> A
    m5-extract-indexed  :  M5 A l -> Indexed A l
    m5-strictly-smaller :  M5 A l -> Vec (Indexed A l) 2
    m5-strictly-larger  :  M5 A l -> Vec (Indexed A l) 2
    m5-extract-l        :  Vec (M5 A l') l
                        -> Vec (Indexed A l') (3 * l)
    m5-extract-r        :  Vec (M5 A l') l
                        -> Vec (Indexed A l') (3 * l)
    m5-extract-strictly-smaller :  Vec (M5 A l') l
                                -> Vec (Indexed A l') (2 * l)
    m5-extract-strictly-larger  :  Vec (M5 A l') l
                                -> Vec (Indexed A l') (2 * l)

    ix-half : Vec A (suc l-1) -> Fin (suc l-1)
    ix-half _ = frombNat< {m = \lfloor suc l-1 /2\rfloor} (n>0=>\lfloorn/2\rfloor<n _)

    simplify-med-split : let l = 5 + l-5 in
                   Ordselect (M5 A l) \lfloor l /5\rfloor (ix-half v)
                -> Split (M5 A l) \lfloor l /5\rfloor
                         (_\equiv \lfloor \lfloor l /5\rfloor /2\rfloor)
                         (_\equiv \lfloor \lfloor l-5 /5\rfloor /2\rfloor)
    simplify-med-split s = let l = 5 + l-5 in
        subst2 (\lambda  lower-bound upper-bound
                -> Split (M5 A l) \lfloor l /5\rfloor
                         (_\equiv lower-bound)
                         (_\equiv upper-bound))
            (tobNat-frombNat< $ n>0=>\lfloorn/2\rfloor<n \lfloor l-5 /5\rfloor)
            (nbNat-bNatsuc\lfloorn/2\rfloor\equiv\lfloorn-1/2\rfloor \lfloor l-5 /5\rfloor)
        s

    small :  let l = 5 + l-5 in
             Split (M5 A l) \lfloor l /5\rfloor
                   (_\equiv \lfloor \lfloor l /5\rfloor /2\rfloor)
                   (_\equiv \lfloor \lfloor l-5 /5\rfloor /2\rfloor)
          -> Vec (Indexed A l) (2 + 3 * (l / 10))
    small s = subst (\lambda k -> Vec _ (2 + 3 * k))
                  (trans (Split.bound-smaller s)
                         (\lfloorn/5/2\rfloor\equivn/10 $ 5 + l-5)) $
              (m5-strictly-smaller $
                 Indexed.value $ Split.median s) ++
              (m5-extract-l $
                 Data.Vec.map Indexed.value $ Split.smaller s)

    large :  let l = 5 + l-5 in
             Split (M5 A l) \lfloor l /5\rfloor
                   (_\equiv \lfloor \lfloor l /5\rfloor /2\rfloor)
                   (_\equiv \lfloor \lfloor l-5 /5\rfloor /2\rfloor)
          -> Vec (Indexed A l) (2 + 3 * (l-5 / 10))
    large s = subst (\lambda k -> Vec _ (2 + 3 * k))
                  (trans (Split.bound-larger s)
                         (\lfloorn/5/2\rfloor\equivn/10 l-5)) $
              (m5-strictly-larger $
                 Indexed.value $ Split.median s) ++
              (m5-extract-r $
                 Data.Vec.map Indexed.value $ Split.larger s)

    unk :  let l = 5 + l-5 in
           Split (M5 A l) \lfloor l /5\rfloor
                 (_\equiv \lfloor \lfloor l /5\rfloor /2\rfloor)
                 (_\equiv \lfloor \lfloor l-5 /5\rfloor /2\rfloor)
        -> Vec (Indexed A l) (2 * (l / 10) + 2 * (l-5 / 10))
    unk s = let l = 5 + l-5 in
        subst2 (\lambda x y -> Vec _ (2 * x + 2 * y))
            (trans (Split.bound-smaller s)
                   (\lfloorn/5/2\rfloor\equivn/10 l))
            (trans (Split.bound-larger s)
                   (\lfloorn/5/2\rfloor\equivn/10 l-5)) $
        (m5-extract-strictly-larger $
           Data.Vec.map Indexed.value $ Split.smaller s) ++
        (m5-extract-strictly-smaller $
           Data.Vec.map Indexed.value $ Split.larger s)
\end{lstlisting}

Here we finally come to (one half of) the meat of our order statistic algorithm. For our quasi-median function, we first group our vector in ordered sets of five. We then select the \emph{actual} median of these sets, specifying that we want to order them by their middle value.

Selecting this median takes time $35 \lfloor \frac l 5 \rfloor$. We cast this in terms of $l$ by showing that $35 \lfloor \frac l 5 \rfloor \leq 7 l$.

Since the bounds for the size of the smaller and the larger section of quickselect's split are expressed in terms of the \texttt{Fin} number type, we reframe this in the \texttt{Nat} type by showing that the bounds are equivalent to $\lfloor \lfloor \frac l 5 \rfloor / 2 \rfloor$ and $\lfloor \lfloor \frac {l - 5} 5 \rfloor / 2 \rfloor$, respectively.

Finally, we extract the part that we are \emph{sure} is smaller than the selected median-of-medians (the "top left" section of the diagram above) and similarly for the larger section. Everything else, including the leftover elements that we couldn't group into sets of five, end up in the unknown block.

What remains is largely mathematical proof burden to show that all of our bounds hold.

\begin{lstlisting}[caption={Quickselect},label={lst:median:quickselect},emph={ordselect,by}]
ordselect-by :  let l = suc l-1 in
                Acc _<_ l
             -> (X -> A)
             -> (i : Fin l)
             -> (xs : Vec X l)
             -> DecTree A (Ordselect X l i) (35 * l)

ordselect-by wf-acc f i xs with \leq-total (suc l-1) 8000
\end{lstlisting}

Finally, we come to an algorithmic sleight of hand: We can only prove that our quickselect bound holds when the length of the input is larger than 8000 \todo{recall why this is the case again}. However, for $l = 8000$, $\lceil \log_2 l \rceil$ is just 13 -- well within our bound! We therefore fall back to the sorting approach for this case, using our linear approach only where we can show it holds.

\begin{lstlisting}[caption={Quickselect ($l \leq 8000$)},label={lst:median:quickselect:small},emph={ordselect,by,split,sorted,merge,sort}]
ordselect-by wf-acc f i xs with \leq-total (suc l-1) 8000
... | inj\_1 l\leq8000 = delay-\leq nlogn\leq35n $
                    split-sorted i <$> sorted
  where
    open \leq-Reasoning
    open import Nat.Log
    nlogn\leq35n : l * \lceillog\_2 l \rceil \leq 35 * l
    nlogn\leq35n = begin
        l * \lceillog\_2 l \rceil  \leq\langle *-mono\^r-\leq l $
                          n\leq8000=>\lceillog\_2n\rceil\leq35 l\leq8000 \rangle
        l * 35         \equiv\langle *-comm l 35 \rangle
        35 * l        \qed
    sorted = merge-sort-by (f \circ Indexed.value) $
             zipWith index (allFin l) xs
\end{lstlisting}

We index all the elements in the list, sort it and then split it at the specified location. Then we simply have to show that sorting the list is indeed faster than $35l$.

\begin{lstlisting}[caption={Quickselect ($l > 8000$)},label={lst:median:quickselect:large},emph={ordselect,by}]
ordselect-by wf-acc f i xs with \leq-total (suc l-1) 8000
... | inj\_2 l>8000 =
    let ibNat = tobNat i in
    height-\equiv (sym $ *-distrib\^r-+ l 9 26) $ do
        split <- quasi-median-by wf-acc f xs
        let l\_1 = Split.l\_1 split
        let l\_2 = Split.l\_2 split
        let l\_3 = Split.l\_3 split
        let median = Split.median split
        let unknown = Split.unknown split
        let l\_3\leql = \leq-trans
                     (m\leqn+m l\_3 (1 + l\_1 + l\_2))
                     (\leq-reflexive $ Split.length-\equiv split)
        unk-smaller , unk-larger by us+ul\equivl₃ <-
            delay-\leq l\_3\leql $
            split-pivot-by (f \circ Indexed.value) median unknown
        let smaller = Split.smaller split ++ unk-smaller
        let larger  = Split.larger  split ++ unk-larger
        let us\leql\_3 = \leq-trans
                      (m\leqm+n (len unk-smaller) (len unk-larger))
                      (\leq-reflexive us+ul\equivl\_3)
        let ul\leql\_3 = \leq-trans
                      (m\leqn+m (len unk-larger) (len unk-smaller))
                      (\leq-reflexive us+ul\equivl\_3)

        case ord ibNat (len smaller) of \lambda where
            (lt pf) -> delay-\leq (smallerRuntimeBound
                                    split
                                    (len unk-smaller)
                                    us\leql\_3) $
                       ordselect-lt
                           wf-acc f i
                           (Split.median split)
                           smaller larger
                           us+ul\equivl\_3
                           (Split.length-\equiv split)
                           pf
            (eq pf) -> delay-\leq z\leqn $ return $
                       ordselect-eq
                           i
                           (Split.median split)
                           smaller larger
                           us+ul\equivl\_3
                           (Split.length-\equiv split)
                           pf
            (gt pf) -> delay-\leq (largerRuntimeBound
                                    split
                                    (len unk-larger)
                                    ul\leql\_3) $
                       ordselect-gt
                           wf-acc f i
                           (Split.median split)
                           smaller larger
                           us+ul\equivl\_3
                           (Split.length-\equiv split)
                           pf
\end{lstlisting}

We first select the quasi median of the provided data in time $9l$. We then further split the unknown data set into a larger and a smaller part with respect to this quasi median in time $l$. Finally, we see how the length of the smaller data set compares to the index we are looking for: If the index is smaller, we just select with the same index from the smaller data set. If the index is equal, we simply return the selected quasi median. And if the index is larger, we subtract the size of the smaller data set from the index and use this to select from the larger data set.

As for the run time, we show that selecting the median from the smaller or larger data set, which runs in time $35 \abs{smaller}$ or $35 \abs{larger}$ respectively, is faster than $25l$, which in addition to the quasi median selection and the splitting of the unknown section adds up to at total run time of $35l$.

Let us inspect this proof for the smaller case. The larger case follows by symmetry.

From the selection of the quasi-median, we know that

\begin{align*}
    1 + \abs{smaller_q} + \abs{larger_q} + \abs{unknown} &= l\\
    1 + \abs{larger_q} &\geq 3 \lfloor \frac l {10} \rfloor.
\end{align*}

We split the unknown set into a smaller and a larger set and add it to the smaller and larger set of the quasi median and thus have

\begin{align*}
    \abs{smaller} &= \abs{smaller_q} + \abs{smaller_u} \\
    \abs{larger} &= \abs{larger_q} + \abs{larger_u} \\
    1 + \abs{smaller} + \abs{larger} &= l
\end{align*}

and thus by transitivity of $\leq$

\begin{align*}
    1 + \abs{larger} &\geq 3 \lfloor \frac l {10} \rfloor.
\end{align*}

Now we can show that

\begin{align*}
    && \abs{smaller} \\
    &=& l - (1 + \abs{larger} \\
    &\leq& l - 3 \lfloor \frac l {10} \rfloor \\
    &=& l \bmod 10 + 10 \lfloor \frac l {10} \rfloor - 3 \lfloor \frac l {10} \rfloor \\
    &=& l \bmod 10 + 7 \lfloor \frac l {10} \rfloor \\
    &\leq& 7 \lfloor \frac l {10} \rfloor + 9.
\end{align*}

Under the assumption that $l \geq 8000$, we can further generalize this to

\begin{align*}
    && 7 \lfloor \frac l {10} \rfloor + 9 \\
    &=& 7 \lfloor \frac {10l} {100} \rfloor + 9 \\
    &\leq& 70 \lfloor \frac {l} {100} \rfloor + 79 \\
    &\leq& 70 \lfloor \frac {l} {100} \rfloor + \lfloor \frac {8000} {100} \rfloor \\
    &\leq& 70 \lfloor \frac {l} {100} \rfloor + \lfloor \frac {l} {100} \rfloor \\
    &=& 71 \lfloor \frac {l} {100} \rfloor.
\end{align*}

And thus selecting an order statistic from the smaller set takes time

\begin{align*}
    && 35 \cdot 71 \cdot \lfloor \frac {l} {100} \rfloor \\
    &\leq& 2500 \lfloor \frac {l} {100} \rfloor \\
    &\leq& 25 l.
\end{align*}